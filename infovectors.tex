void VisualGeometryObject::УстановитьФорматВершинИнфовекторов()
{

	TechniqueInfovectors= fx->GetTechniqueByName("InfovecTech");

	D3D10_INPUT_ELEMENT_DESC vertexformat[] = //массив элементов, описывающих вершину
	{

	// 1) Имя семантики в шейдере, 2) индекс семантики, 3) формат компонента вершины (Position - vector3f  [3*float = 3*32 bits = 12 bytes]), 
	// 4) слот (индекс слота) через к-ый будет передаваться данный компонент вершины (всего 16 слотов с индексами 0-15), 
	// 5) смещение в байтах для данного компонента вершины (от начала массива до начала данного компонента вершины)
		
		//   1       2           3                      4  5
		{"POSITION", 2, DXGI_FORMAT_R32G32B32_FLOAT,    0, 0,   D3D10_INPUT_PER_VERTEX_DATA, 0},
		{"COLOR",    2, DXGI_FORMAT_R32G32B32A32_FLOAT, 0, 3*4, D3D10_INPUT_PER_VERTEX_DATA, 0},

	};



	D3D10_PASS_DESC PassDesc;

	
	TechniqueInfovectors->GetPassByIndex(0)->GetDesc(&PassDesc);
	d3dDevice->CreateInputLayout(vertexformat, 2, PassDesc.pIAInputSignature, PassDesc.IAInputSignatureSize, &VertexLayoutInfovectors);

}


void VisualGeometryObject::СоздатьИнфовекторы(const vector<VertexPosColor>* vertices, const vector<DWORD>* indices)
{

NumInfovectorsVertices= vertices->size();


D3D10_BUFFER_DESC vbdesc;

	vbdesc.ByteWidth= sizeof(VertexPosColor)*NumInfovectorsVertices;
	//vbdesc.Usage= D3D10_USAGE_IMMUTABLE;
	vbdesc.Usage= D3D10_USAGE_DYNAMIC;
	vbdesc.BindFlags= D3D10_BIND_VERTEX_BUFFER;
	//vbdesc.CPUAccessFlags= 0;
	vbdesc.CPUAccessFlags= D3D10_CPU_ACCESS_WRITE;
	vbdesc.MiscFlags= 0;

	D3D10_SUBRESOURCE_DATA vsbd;
	vsbd.pSysMem= &((*vertices)[0]);
	d3dDevice->CreateBuffer(&vbdesc, &vsbd, &VB_infovectors);



D3D10_BUFFER_DESC ibdesc;

	ibdesc.ByteWidth= sizeof(DWORD)*NumInfovectorsVertices;
	ibdesc.Usage= D3D10_USAGE_IMMUTABLE;
	ibdesc.BindFlags= D3D10_BIND_INDEX_BUFFER;
	ibdesc.CPUAccessFlags= 0;
	ibdesc.MiscFlags= 0;

	D3D10_SUBRESOURCE_DATA isbd;

	isbd.pSysMem= &((*indices)[0]);
	d3dDevice->CreateBuffer(&ibdesc, &isbd, &IB_infovectors);


}



//void VisualGeometryObject::ОтобразитьИнфовекторы(const vector<VertexPosColor>* vertices)
//void VisualGeometryObject::ОтобразитьИнфовекторы(const D3DXVECTOR3 vector, const D3DXMATRIX m)
void VisualGeometryObject::ОтобразитьИнфовекторы(const vector<VertexPosColor>& vectors, const D3DXMATRIX m)
{

	
	D3DXVECTOR3 v3(2, 2, 2), vOut;

	D3DXMATRIX m44;

	float s= (float)1/2;

	D3DXMatrixIdentity(&m44);

	m44._11*= s;	m44._22*= s;	m44._33*= s;
	m44._41= 1;		m44._42= 1;		m44._43= 1;


	vOut= Vec3TransformAsPoint(v3, m44);
	
	
	//Обновляем вершинный буфер инфовекторов

		VertexPosColor* v= 0;

		


	//VB_infovectors->Map(D3D10_MAP_WRITE_DISCARD, 0, (void**) &((*vertices)[0]));
	VB_infovectors->Map(D3D10_MAP_WRITE_DISCARD, 0, (void**) &v);

	
		for(DWORD i= 0; i < vectors.size(); i++)
		{
		
			//v[i].pos= Vec3TransformAsPoint(vectors[i].pos, m);
			v[i].pos= vectors[i].pos;
			v[i].color= vectors[i].color;
		
		}

		/*v[0].pos= vectors[0].pos;
		v[0].color= vectors[0].color;

		v[1].pos= vectors[1].pos;
		v[1].color= vectors[1].color;

		v[2].pos= vectors[2].pos;
		v[2].color= vectors[2].color;
		
		v[3].pos= vectors[3].pos;
		v[3].color= vectors[3].color;*/


	VB_infovectors->Unmap();
	
	
	// Restore default states, input layout and primitive topology we can restore the default states by passing null

	d3dDevice->OMSetDepthStencilState(0, 0);
	float blendFactors[] = {0.0f, 0.0f, 0.0f, 0.0f};
	d3dDevice->OMSetBlendState(0, blendFactors, 0xffffffff);
	d3dDevice->IASetInputLayout(VertexLayoutInfovectors);
	//d3dDevice->IASetPrimitiveTopology(D3D10_PRIMITIVE_TOPOLOGY_POINTLIST);

	d3dDevice->IASetPrimitiveTopology(D3D10_PRIMITIVE_TOPOLOGY_LINELIST);


	//LWVP= ScaleMatrix*View;
	//LWVP= View;

	fxLWVP->SetMatrix((float*)&LWVP);

	//------------------- Отрисовка -- DrawIndexed() ---------------------//


	UINT stride= sizeof(VertexPosColor); // массив страйдов для каждого ВБ, где значение элемента равно
										 // размеру в байтах элемента соответствующего вершинного буфера;
									 	 // если ВБ один, то массив страйдов вырождается до скаляра;
										 // в метод передаётся указатель на первый элемент массива страйдов

	UINT offset= 0; // массив смещений для каждого ВБ, где значение элемента равно
					// размеру смещения в байтах от начала соответствующего вершинного буфера;
					// если ВБ один, то массив смещений вырождается до скаляра;
					// в метод передаётся указатель на первый элемент массива офсетов;

					// смещение - количество байт от начала ВБ до элемента, с которого сборщик (IA) начнёт читать данные;
					// имеет смысл тогда, когда одним ВБ задаётся сразу несколько объектов, тогда для каждого нужно задать смещение


	D3D10_TECHNIQUE_DESC d3dtechdesc;

	TechniqueInfovectors->GetDesc(&d3dtechdesc);

    for(UINT i = 0; i < d3dtechdesc.Passes; ++i)
    {

        //ID3D10EffectPass* pass = mTech->GetPassByIndex(i);
		
		TechniqueInfovectors->GetPassByIndex( i )->Apply(0); // Передаём установленные выше значения в соответсвующие константные буферы шейдера

		d3dDevice->IASetVertexBuffers(0, 1, &VB_infovectors, &stride, &offset);
		d3dDevice->IASetIndexBuffer(IB_infovectors, DXGI_FORMAT_R32_UINT, 0); // DXGI_FORMAT_R32_UINT - т.к. для индексов был использован тип DWORD

		d3dDevice->DrawIndexed(NumInfovectorsVertices, 0, 0);

    }

	

}